%----------------------------------------------------------------------------------------
%	CONFIGURACION DOCUMENTO GENERAL
%----------------------------------------------------------------------------------------

\documentclass[12pt,a4paper,oneside]{report} % Formato tesis, 12 tamaño de letra
\usepackage[english]{babel} % Asignar idioma por defecto spanish
%\selectlanguage{spanish} % Idioma por defecto x2
\usepackage[T1]{fontenc} % Aceptar caracteres con tíldes en el pdf generado
\usepackage[UTF8]{ctex}
\usepackage[utf8]{inputenc} % Aceptar caracteres con tíldes en Latex
\usepackage{amsmath} % Para mostrar de forma adecuada las ecuaciones
\usepackage{graphicx} % Para incluir imágenes 
\usepackage{array} % Para incluir imagenes en tablas https://bit.ly/2KXhIlu
\usepackage{float} % Para posicionar imágenes y tablas
\usepackage{multirow} % Fácil manejo de filas
\usepackage{longtable} % Para tablas que abarcan más de dos páginas
\usepackage[table,xcdraw]{xcolor} % Para tablas generadas en https://www.tablesgenerator.com/
% Tamaños de letra dentro de tablas https://texblog.org/2012/08/29/changing-the-font-size-in-latex/
\usepackage{multicol} % Para poder partir en columnas 
\usepackage[refpages]{gloss} % Para enumerar páginas
\usepackage{anysize} % OBSOLETO, NECESITA SER CAMBIADO
\usepackage{bigstrut} % Para automatizar ingresado de datos en tabla
\usepackage{appendix} % Formato para títulos de los apéndices
\usepackage{lscape} % Modifica margenes y rota la página pero no el enumerado
\usepackage{pdflscape} % Cambiar orientación en el pdf
\usepackage{listings} % Sirve para que se compile el Latex
\usepackage{color} % Colores de fondo, texto y otros
\usepackage{setspace} % Espacio entre lineas
\usepackage{enumerate} % Estilo del contador de páginas
\usepackage{ragged2e} % Formato de parrafos, justiticar y demás
\usepackage{comment} % Hacer comentarios en latex
\usepackage{pslatex} % Tipo de fuente
\usepackage{apacite} % Citación en APA
\usepackage{fixltx2e} % Corrige bugs de Latex 3
\usepackage{caption} % Objetos flotantes en el documento
\usepackage[top=2.54cm,
			bottom=2.54cm,
			left=2.54cm,
			right=2.54cm,
			headheight=17pt, % as per the warning by fancyhdr
			includehead,includefoot,
			heightrounded, % to avoid spurious underfull messages
			]{geometry} % Márgenes para contenido y líneas de pie y cabecera de página
\usepackage{fancyhdr} % Cabeza y pie de página 
\usepackage{blindtext}
\usepackage{pdfpages} % Cargar imagenes en pdf
\usepackage{datetime} % Fecha
\usepackage{footnote} % Tablas largas y foot note para tablas https://texblog.org/2012/02/03/using-footnote-in-a-table/
\usepackage{enumitem} % Enumerar letras
\usepackage[bottom]{footmisc} % Forzar notas de página pegadas hacia abajo https://bit.ly/3fyOA2f


\usepackage[paper=portrait,pagesize]{typearea} %PORTRAID-LANDSCAPE 

\usepackage{url} % Evitar enlaces en bibliografía largos & enlazar enlaces de bibliografía
\usepackage{breakurl} % Broken links
\usepackage{hyperref} % Evitar enlaces en bibliografía largos & enlazar enlaces de bibliografía
\usepackage{natbib}
\usepackage{etoolbox}

%----------------------------------------------------------------------------------------
%	CONFIGURACION FORMATO SECCIONES
%----------------------------------------------------------------------------------------

\fancypagestyle{blank-page}{
	\fancyhf{}% clear all header and footer fields	
	\fancyhead[L,C,R]{}
	\fancyfoot[L,C,R]{}
}

\fancypagestyle{after-table-of-content}{
	\fancyhf{}% clear all header and footer fields	
	\fancyhead[L,C,R]{}
	\fancyfoot[L,C,R]{}
	\fancyfoot[c]{PUCP}
	\renewcommand{\headrulewidth}{0.4pt}
	\renewcommand{\footrulewidth}{0.4pt}
}
\fancypagestyle{myportland}{
	\fancyhf{}% clear all header and footer fields	
	\fancyhead[L,C,R]{}
	\fancyfoot[L,C,R]{}
	\fancyfoot[c]{\thepage}
	\renewcommand{\headrulewidth}{0.4pt}
	\renewcommand{\footrulewidth}{0.4pt}
}
% https://tex.stackexchange.com/questions/337/how-to-change-certain-pages-into-landscape-portrait-mode
\fancypagestyle{mylandscape}{ % LANDSCAPE	
	\fancyhead{}% clear all header 
	\fancyfoot{}% clear all footer
	\fancyhf{}% clear all header and footer fields	
	\fancyhead[LO,RE]{}
	\fancyfoot[LO,RE]{}
	\fancyfoot{% Footer
		\makebox[\textwidth][r]{% Right
			\rlap{\hspace{0.75cm}% Push out of margin by \footskip
				\smash{% Remove vertical height
					\raisebox{4.87in}{% Raise vertically
						\rotatebox{90}{\thepage}}}}}}% Rotate counter-clockwise
	\renewcommand{\headrulewidth}{0pt}% No header rule
	\renewcommand{\footrulewidth}{0pt}% No footer rule
}


\newcommand*\rot{\rotatebox{90}} % Poder rotar 90° texto, imagenes, lo que sea
\graphicspath{{images/}} % Directorio que contiene imagenes
\captionsetup[table]{skip=10pt} % Posición tablas
\captionsetup[figure]{skip=10pt} % Posición figuras
\captionsetup{justification=centering}


%----------------------------------------------------------------------------------------
%	CONFIGURACION NOMBRE DE TÍTULOS, CAPÍTULOS, TABLAS, ÍNDICES
%----------------------------------------------------------------------------------------
\addto\captionsenglish{
\renewcommand{\contentsname}{Índice}
\renewcommand{\listfigurename}{Índice de figuras}
\renewcommand{\listtablename}{Índice de tablas}
\renewcommand{\baselinestretch}{1.0}
\renewcommand{\appendixname}{Anexos}
\renewcommand{\appendixtocname}{Anexos}
\renewcommand{\appendixpagename}{Anexos}
\renewcommand{\chaptername}{Capítulo}
\renewcommand{\tablename}{Tabla}
\renewcommand{\figurename}{Figura}
\renewcommand{\thechapter}{\Roman{chapter}}
\renewcommand{\thesection}{\arabic{chapter}.\arabic{section}}
\renewcommand{\thetable}{\arabic{chapter}.\arabic{table}}
\renewcommand{\thefigure}{\arabic{chapter}.\arabic{figure}}
\renewcommand{\theequation}{\arabic{chapter}.\arabic{equation}}
\renewcommand{\bibname}{Bibliografía}
\renewcommand{\refname}{Bibliografía}
}


%----------------------------------------------------------------------------------------
%	CONFIGURACION BIBLIOGRAFIA
%----------------------------------------------------------------------------------------
\bibliographystyle{apacite}
\renewcommand{\BOthers}[1]{et al.\hbox{}}%       et al
\renewcommand{\BOthersPeriod}[1]{et al.\hbox{}}%  et al.
