%----------------------------------------------------------------------------------------
%	AUTOR DE LA CONFIGURACIÓN ACTUAL
%----------------------------------------------------------------------------------------

% Pablo Díaz Vergara
% Contáctamen en: pablo.diazv@pucp.edu.pe
% Revisa las últimas actualizaciones en:
% https://bit.ly/2EV6luX
% Versión Cofiguraciones: v2.0
% Fecha: 2020-09-04
% Github: https://github.com/ZurMaD/plantilla_tesis_pucp/

% AVISO: No modificar por su cuenta, enviar un correo para arreglar errores!

%----------------------------------------------------------------------------------------
%	NOMBRE DEL ARCHIVO
%----------------------------------------------------------------------------------------

% Eliminado por bugs en la visualización

%----------------------------------------------------------------------------------------
%	CONFIGURACION DOCUMENTO GENERAL
%----------------------------------------------------------------------------------------

\documentclass[12pt,a4paper,oneside]{thesis} % Formato tesis, 12 tamaño de letra
\usepackage[english]{babel} % Asignar idioma por defecto spanish
%\selectlanguage{spanish} % Idioma por defecto x2
\usepackage[T1]{fontenc} % Aceptar caracteres con tíldes en el pdf generado
\usepackage[UTF8]{ctex}
\usepackage[utf8]{inputenc} % Aceptar caracteres con tíldes en Latex
\usepackage{amsmath} % Para mostrar de forma adecuada las ecuaciones
\usepackage{graphicx} % Para incluir imágenes 
\usepackage{array} % Para incluir imagenes en tablas https://bit.ly/2KXhIlu
\usepackage{float} % Para posicionar imágenes y tablas
\usepackage{multirow} % Fácil manejo de filas
\usepackage{longtable} % Para tablas que abarcan más de dos páginas
\usepackage[table,xcdraw]{xcolor} % Para tablas generadas en https://www.tablesgenerator.com/
% Tamaños de letra dentro de tablas https://texblog.org/2012/08/29/changing-the-font-size-in-latex/
\usepackage{multicol} % Para poder partir en columnas 
\usepackage[refpages]{gloss} % Para enumerar páginas
\usepackage{anysize} % OBSOLETO, NECESITA SER CAMBIADO
\usepackage{bigstrut} % Para automatizar ingresado de datos en tabla
\usepackage{appendix} % Formato para títulos de los apéndices
\usepackage{lscape} % Modifica margenes y rota la página pero no el enumerado
\usepackage{pdflscape} % Cambiar orientación en el pdf
\usepackage{listings} % Sirve para que se compile el Latex
\usepackage{color} % Colores de fondo, texto y otros
\usepackage{setspace} % Espacio entre lineas
\usepackage{enumerate} % Estilo del contador de páginas
\usepackage{ragged2e} % Formato de parrafos, justiticar y demás
\usepackage{comment} % Hacer comentarios en latex
\usepackage{pslatex} % Tipo de fuente
\usepackage{apacite} % Citación en APA
\usepackage{fixltx2e} % Corrige bugs de Latex 3
\usepackage{caption} % Objetos flotantes en el documento
\usepackage[top=2.54cm,
bottom=2.54cm,
left=2.54cm,
right=2.54cm,
headheight=17pt, % as per the warning by fancyhdr
includehead,includefoot,
heightrounded, % to avoid spurious underfull messages
]{geometry} % Márgenes para contenido y líneas de pie y cabecera de página
\usepackage{fancyhdr} % Cabeza y pie de página 
\usepackage{blindtext}
\usepackage{pdfpages} % Cargar imagenes en pdf
\usepackage{datetime} % Fecha
\usepackage{footnote} % Tablas largas y foot note para tablas https://texblog.org/2012/02/03/using-footnote-in-a-table/
\usepackage{enumitem} % Enumerar letras
\usepackage[bottom]{footmisc} % Forzar notas de página pegadas hacia abajo https://bit.ly/3fyOA2f

\usepackage[paper=portrait,pagesize]{typearea} %PORTRAID-LANDSCAPE 

\usepackage{url} % Evitar enlaces en bibliografía largos & enlazar enlaces de bibliografía
\usepackage{breakurl} % Broken links
\usepackage{hyperref} % Evitar enlaces en bibliografía largos & enlazar enlaces de bibliografía
\usepackage{natbib}
\usepackage{etoolbox}

\usepackage{mathptmx} % Times News Roman
\usepackage{titlesec} % Cambiar chapter style
\usepackage{indentfirst} % Agregar indentado de 1cm al inicio de cada párrafo de cada sección


%----------------------------------------------------------------------------------------
%	CONFIGURACION ESPACIADO, TAMAÑO DE LETRA Y RELACIONADOS A FIGURAS, TABLAS
%----------------------------------------------------------------------------------------
% Documentación \captionsetup https://bit.ly/31RjwG3 

% FIGURAS
\captionsetup[figure]{font={small,singlespacing}} % https://bit.ly/2QOHo6U
\captionsetup[figure]{labelfont={bf}} % Negrita a Tabla X.X
\captionsetup[figure]{labelsep={endash}} % Punto y espacio luego de Figura X.X. 
\captionsetup[figure]{width={.75\textwidth}} % Sangría para título de Figura
\captionsetup[figure]{justification={justified,centerlast}} % Justificado, última linea centrado
\captionsetup[figure]{aboveskip=7pt} % https://bit.ly/3jIe7HF Foro latex
\captionsetup[figure]{belowskip=7pt} 
\captionsetup[figure]{skip=0pt} % Posición figuras
\captionsetup[figure]{position=below} % Posición figuras
\captionsetup[figure]{name=Figura} % Nombre figuras

% TABLAS
\captionsetup[table]{font={small,singlespacing}} % scriptsize,footnotesize,small,normalsize,larga,Large %stretch=0.8
\captionsetup[table]{labelfont={bf}} % normalfont,up,it,sl,sc,md,bf,rm,sf,tt
\captionsetup[table]{labelsep={endash}} % none,colon,period,space,quad,newline,endash 
\captionsetup[table]{width={.75\textwidth}} % Sangría para título de tabla
\captionsetup[table]{justification={justified,centerlast}} % Justificado, última linea centrado
\captionsetup[table]{aboveskip=7pt} % Espaciado arriba del título
\captionsetup[table]{belowskip=7pt} % Espaciado abajo del título
\captionsetup[table]{skip=0pt} % Posición tablas
\captionsetup[table]{position=above} % Posición tablas
\captionsetup[table]{name=Tabla} % Nombre figuras

%----------------------------------------------------------------------------------------
%	CONFIGURACION SECCIONES DE CÓDIGO
%----------------------------------------------------------------------------------------

% Eliminado, contenia bugs, la siguiente versión se implementa

%----------------------------------------------------------------------------------------
%	CONFIGURACION FORMATO SECCIONES
%----------------------------------------------------------------------------------------

%%%%%%%%%%%%%%%%%%%%%%%%%%%%%%%%%%%%%%%%%%%%%%%%%%%%%%
% PÁGINA DE CAPÍTULOS ES AUTOMÁTICO TIPO PLAIN       %
%%%%%%%%%%%%%%%%%%%%%%%%%%%%%%%%%%%%%%%%%%%%%%%%%%%%%%
\fancypagestyle{plain}{
	\fancyhead{}% clear all header 
	\fancyfoot{}% clear all footer
	\fancyhf{}% clear all header and footer fields	
	\fancyhead[L,C,R]{}
	\fancyfoot[L,C,R]{}
	\fancyfoot[C]{Página \thepage}
	\renewcommand{\headrulewidth}{0.4pt}
	\renewcommand{\footrulewidth}{0.4pt}
}
%%%%%%%%%%%%%%%%%%%%%%%%%%%%%%%%%%%%%%%%%%%%%%%%%%%%%%
% PÁGINA VERTICAL                                    %
%%%%%%%%%%%%%%%%%%%%%%%%%%%%%%%%%%%%%%%%%%%%%%%%%%%%%%
\fancypagestyle{myportland}{
	\fancyhead{}% clear all header 
	\fancyfoot{}% clear all footer
	\fancyhf{}% clear all header and footer fields	
	\fancyhead[L,C,R]{}
	\fancyfoot[L,C,R]{}
	\fancyfoot[C]{Página \thepage}
	\renewcommand{\headrulewidth}{0.4pt}
	\renewcommand{\footrulewidth}{0.4pt}
}
%%%%%%%%%%%%%%%%%%%%%%%%%%%%%%%%%%%%%%%%%%%%%%%%%%%%%%
% PÁGINA HORIZONTAL                                  %
%%%%%%%%%%%%%%%%%%%%%%%%%%%%%%%%%%%%%%%%%%%%%%%%%%%%%%
% https://tex.stackexchange.com/questions/337/how-to-change-certain-pages-into-landscape-portrait-mode
\fancypagestyle{mylandscape}{
	\fancyhead{}% clear all header 
	\fancyfoot{}% clear all footer
	\fancyhf{}% clear all header and footer fields	
	\fancyhead[LO,RE]{}
	\fancyfoot[LO,RE]{}
	\fancyfoot{% Footer
		\makebox[\textwidth][r]{% Right
			\rlap{\hspace{0.75cm}% Push out of margin by \footskip
				\smash{% Remove vertical height
					\raisebox{4.87in}{% Raise vertically
						\rotatebox{90}{Página \thepage}}}}}}% Rotate counter-clockwise
	\renewcommand{\headrulewidth}{0pt}% No header rule
	\renewcommand{\footrulewidth}{0pt}% No footer rule
}


%----------------------------------------------------------------------------------------
%	UTILIDADES
%----------------------------------------------------------------------------------------

\newcommand*\rot{\rotatebox{90}} % Poder rotar 90° texto, imagenes, lo que sea
\graphicspath{{images/}} % Directorio que contiene imagenes


%----------------------------------------------------------------------------------------
%	CONFIGURACION NOMBRE DE TÍTULOS, CAPÍTULOS, TABLAS, ÍNDICES
%----------------------------------------------------------------------------------------


%%%%%%%%%%%%%%%%%%%%%%%%%%%%%%%%%%%%%%%%%%%%%%%%%%%%%%
% CAMBIO DE FORMATO DE CAPÍTULOS                     %
%%%%%%%%%%%%%%%%%%%%%%%%%%%%%%%%%%%%%%%%%%%%%%%%%%%%%%

% CHAPTER X
\titleformat{\chapter}
{\large\bfseries\filcenter}{\thechapter}{1em}{}
% SECTION X.X
\titleformat{\section}
{\normalfont\bfseries\filright}{\thesection}{1em}{}

% SUBSECTION X.X.X
\titleformat{\subsection}
{\normalfont\bfseries\filright}{\thesubsection}{1em}{}

% SUBSUBSECTION X.X.X.X
\titleformat{\subsubsection}
{\normalfont\bfseries\filright}{\thesubsubsection}{1em}{}

% Dentro del siguiente código no usar comentarios
\addto\captionsenglish{
	\renewcommand{\contentsname}{Índice}
	\renewcommand{\listfigurename}{Índice de figuras}
	\renewcommand{\listtablename}{Índice de tablas}
	\renewcommand{\baselinestretch}{2}
	\renewcommand{\appendixname}{Anexos}
	\renewcommand{\appendixtocname}{Anexos}
	\renewcommand{\appendixpagename}{Anexos}
	\renewcommand{\chaptername}{Capítulo}
	\renewcommand{\tablename}{Tabla}
	\renewcommand{\figurename}{Figura}
	\renewcommand{\thechapter}{\Roman{chapter}}
	\renewcommand{\thesection}{\arabic{chapter}.\arabic{section}}
	\renewcommand{\thetable}{\arabic{chapter}.\arabic{table}}
	\renewcommand{\thefigure}{\arabic{chapter}.\arabic{figure}}
	\renewcommand{\theequation}{\arabic{chapter}.\arabic{equation}}
}

\renewcommand{\bibsection}{} % Eliminar "References" título


%%%%%%%%%%%%%%%%%%%%%%%%%%%%%%%%%%%%%%%%%%%%%%%%%%%%%%
% MODO 1: ENUMERACIÓN POR CAPÍTULOS 1.1,1.2,2.1,...  %
%%%%%%%%%%%%%%%%%%%%%%%%%%%%%%%%%%%%%%%%%%%%%%%%%%%%%%
%\renewcommand{\thetable}{\arabic{chapter}.\arabic{table}}
%\renewcommand{\thefigure}{\arabic{chapter}.\arabic{figure}}
%\renewcommand{\theequation}{\arabic{chapter}.\arabic{equation}}
%%%%%%%%%%%%%%%%%%%%%%%%%%%%%%%%%%%%%%%%%%%%%%%%%%%%%%
% MODO 2: ENUMERACIÓN ARÁBICA SECUENCIAL 1,2,3,4,... %
%%%%%%%%%%%%%%%%%%%%%%%%%%%%%%%%%%%%%%%%%%%%%%%%%%%%%%
%\renewcommand{\thetable}{\arabic{1}}
%\renewcommand{\thefigure}{\arabic{1}}
%\renewcommand{\theequation}{\arabic{1}}

%----------------------------------------------------------------------------------------
%	CONFIGURACION BIBLIOGRAFIA
%----------------------------------------------------------------------------------------

\bibliographystyle{apacite}
\renewcommand{\BOthers}[1]{et al.\hbox{}}%       et al
\renewcommand{\BOthersPeriod}[1]{et al.\hbox{}}%  et al.
% CONTINUA EN BIBLIOGRAPHY.TEX